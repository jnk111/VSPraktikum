\documentclass{article}

\usepackage[utf8]{inputenc}
\usepackage[english,german]{babel}
\usepackage{tikz}

\usetikzlibrary{shapes.geometric, arrows, positioning}

\tikzstyle{box}   = [rectangle, minimum width=3cm, minimum height=1cm,text centered, draw=black]
\tikzstyle{arrow} = [thick,->,>=stealth]

\begin{document}

\section{Ausgangssituation:}
\begin{itemize}
    \item Es existiert das Spiel /games/foo
    \item Clientservice \#2 ist bereits dem Spiel begetreten
    \item Das Spiel ist noch nicht gestartet (weitere Spiele können beitreten)
\end{itemize}

\section{Diagramm:}
\begin{tikzpicture}[node distance=5cm]
    \node (client1)      [box]                            {Clientservice \#1};
    \node (gameservice)  [box, right= 5cm of client1]     {Gameservice};
    \node (eventservice) [box, below of=gameservice]      {Eventservice};
    \node (client2)      [box, left= 5cm of eventservice] {Clientservice \#2};
    \node (gui)          [box, below of=client2]          {GUI (Client \#2)};

    \draw [arrow] (client1)      -- node[anchor=south] {Betritt Spiel /games/foo}   (gameservice);
    \draw [arrow] (gameservice)  -- node[anchor=east]  {Erstellt Event über Beitritt}  (eventservice);
    \draw [arrow] (eventservice) -- node[anchor=north] {Informiert über das Event}     (client2);
    \draw [arrow] (client2)      -- node[anchor=west]  {Aktualisiert Spielerliste}     (gui);
\end{tikzpicture}

\end{document}